% *** en embryo of a technical report describing the labcomm design rationale and implementation ***

\documentclass[a4paper]{article}
%\usepackage{verbatims}
%\usepackage{todo}

\begin{document}
\title{Labcomm tech report}
\author{Anders Blomdell and Sven Gesteg\aa{}rd Robertz }
\date{embryo of draft, \today}

\maketitle 

\begin{abstract}

LabComm is a binary protocol suitable for transmitting and storing samples of
process data. It is self-describing and independent of programming language,
processor, and network used (e.g., byte order, etc).  It is primarily intended
for situations where the overhead of communication has to be kept at a minimum,
hence LabComm only requires one-way communication to operate. The one-way
operation also has the added benefit of making LabComm suitable as a storage
format. 

LabComm provides self-describing channels, as communication starts with the
transmission of an encoded description of all possible sample types that can
occur, followed by any number of actual samples in any order the sending
application sees fit.

The LabComm system is based on a binary protocol and
and a compiler that generates encoder/decoder routines for popular languages
including C, Java, and Python. There is also a standard library for the
languages supported by the compiler, containing generic routines for
encoding and decoding types and samples, and interaction with
application code.

The LabComm compiler accepts type and sample declarations in a small language
that is similar to C or Java type-declarations. 
\end{abstract}
\section{Introduction}

%[[http://rfc.net/rfc1057.html|Sun RPC]]
%[[http://asn1.org|ASN1]].

LabComm has got it's inspiration from Sun RPC~\cite{SunRPC}
and ASN1~\cite{ANS1}. LabComm is primarily intended for situations
where the overhead of communication has to be kept at a minimum, hence LabComm
only requires one-way communication to operate. The one-way operation also has
the added benefit of making LabComm suitable as a storage format. 

Two-way comminication adds complexity, in particular for hand-shaking
during establishment of connections, and the LabComm library provides
support for (for instance) avoiding deadlocks during such hand-shaking.

\section{Communication model}

LabComm provides self-describing communication channels, by always transmitting
a machine readable description of the data before actual data is
sent\footnote{Sometimes referred to as \emph{in-band} self-describing}. 
Therefore, communication on a LabComm channel has two phases

\begin{enumerate}
\item the transmission of signatures (an encoded description including data
types and names, see appendix~\ref{sec:ProtocolGrammar} for details) for all sample types
that can be sent on the channel
\item the transmission of any number of actual samples in any order
\end{enumerate}

During operation, LabComm will ensure (i.e., monitor) that a communication
channel is fully configured, meaning that both ends agree on what messages that
may be passed over that channel.  If an unregistered sample type is sent or
received, the LabComm encoder or decoder will detect it and take action.

In more dynamic applications, it is possible to reconfigure a channel in order to add,
remove, or change the set of registerd sample types.

The roles in setting up, and maintaining, the configuration of a channel are as follows:

\paragraph{The application software} (or higher-level protocol) is required to

\begin{itemize}
\item register all samples to be sent on a channel with the encoder
\item register handlers for all samples to be received  on a channel with the decoder
\end{itemize}

\paragraph{The transmitter (encoder)}

\begin{itemize}
 \item ensures that the signature of a sample is transmitted on the channel before samples are 
       written to that channel
\end{itemize}

\paragraph{The receiver (decoder)}

\begin{itemize}
 \item checks, for each signature, that the application has registered a handler for that sample type
 \item if an unhandled signature is received, pauses the channel and informs the application
\end{itemize}

\subsection{Reconfiguration}

The fundamental communication model can be generalized to the life-cycle
of a concrete communication channel, including the transport layer,
between two end-points. Then, the communication phases are
\begin{enumerate}
  \item \emph{Establishment} of communication channel at the transport layer
  \item \emph{Configuration} of the LabComm channel (registration of sample
    types)
  \item \emph{Operation} (transmission of samples)
\end{enumerate}
where it is possible to \emph{reconfigure} a channel by transitioning
back from phase 3 to phase 2. That allows dynamic behaviour, as a sample
type can be added or redefined at run-time. It also facilitates error
handling in two-way channels.

\section{The Labcomm language}

The LabComm language is used to describe data types, and from such data
descriptions, the compiler generates code for encoding and decoding
samples. The language is quite similar to C struct declarations, with
some exceptions. We will now introduce the language through a set of
examples.

These examples do not cover the entire language
specification (see appendix~\ref{language_grammar} for the complete
grammar), but serve as a gentle introduction to the LabComm
language covering most common use-cases.

\subsection{Primitive types}

\begin{verbatim}
  sample boolean a_boolean;
  sample byte a_byte;
  sample short a_short;
  sample int an_int;
  sample long a_long;
  sample float a_float;
  sample double a_double;
  sample string a_string;
\end{verbatim}

\subsection{Arrays}

\begin{verbatim}
  sample int fixed_array[3];
  sample int variable_array[_];                // Note 1
  sample int fixed_array_of_array[3][4];       // Note 2
  sample int fixed_rectangular_array[3, 4];    // Note 2
  sample int variable_array_of_array[_][_];    // Notes 1 & 2
  sample int variable_rectangular_array[_, _]; // Notes 1 & 2
\end{verbatim}

\begin{enumerate}
\item In contrast to C, LabComm supports both fixed and variable (denoted
by \verb+_+) sized arrays.

\item In contrast to Java, LabComm supports multidimensional arrays and not
only arrays of arrays.

\end{enumerate}

\subsection{Structures}

\begin{verbatim}
  sample struct {
    int an_int_field;
    double a_double_field;
  } a_struct;
\end{verbatim}

\section{User defined types}

\begin{verbatim}
  typedef struct {
    int field_1;
    byte field_2;
  } user_type[_];
  sample user_type a_user_type_instance;
  sample user_type another_user_type_instance;
\end{verbatim}

\section{The LabComm system}

The LabComm system consists of a compiler for generating code from the data
descriptions, and libraries providing LabComm communication facilities in,
currently, C, Java, Python, and C\#.


\subsection{The LabComm compiler}

The LabComm compiler generates code for the declared samples, including marshalling and
demarshalling code, in the supported target languages.

The compiler itself is implemented in Java using the JastAdd~\cite{jastadd} compiler compiler.
\pagebreak
\subsection{The LabComm library}

The LabComm libraries contain functionality for the end-to-end transmission
of samples. They are divided into two layers, where the upper layer implements
the general encoding and decoding of samples, and the lower one deals with 
the transmission of the encoded byte stream on a particular transport layer.

Thus, the LabComm communication stack looks like this:
\begin{figure}[h!]
\begin{verbatim}
    _______________________
    |     Application     |
    +---------------------+
    | encoder  | decoder  |    to/from labcomm encoded byte stream
    +----------+----------+
    | writer   | reader   |    transmit byte stream over particular transport
    +----------+----------+
    | transport layer / OS|
    +---------------------+
\end{verbatim}
\end{figure}
\subsubsection{LabComm actions}

(similar to ioctl()) 
The encoder/writer and decoder/reader interfaces consist of a set of actions

One example of this is that there is a a separate writer action for
transmitting signatures, allowing the writer to treat a signature differently
from encoded samples, e.g., to allow handshaking during channel setup.

User actions allow the application or a higher level
protocol to communicate with the underlying transport layer through the LabComm
encoder. 

One example is reliable communication, which is controlled from the application
but needs to be implemented for each transport at at the reader/writer level.
(Or not, e.g., TCP)

\section{LabComm is not...}

\begin{itemize}
\item a protocol for two-way connections
\item intrinsically supporting reliable communication 
\item providing semantic service-descriptions
\end{itemize}

But

\begin{itemize}
\item it is suitable for the individual channels of a structured connection
\item the user action mechanism allows using feature of different transport layers
  through labcomm (i.e., it allows encapsulation of the transport layer)
\item the names of samples can be chosen and mapped according to a suitable taxonomy or ontology
\end{itemize}



\section{Example and its encoding}

With the following `example.lc` file:

\begin{verbatim}
sample struct {
  int sequence;
  struct {
    boolean last;
    string data;
  } line[_];
} log_message;
sample float data;
\end{verbatim}

and this \verb+example_encoder.c+ file 

\begin{verbatim}
#include <sys/types.h>
#include <sys/stat.h>
#include <fcntl.h>
#include <labcomm_fd_reader.h>
#include <labcomm_fd_writer.h>
#include "example.h"

int main(int argc, char *argv[]) {
  int fd;
  struct labcomm_encoder *encoder;
  int i, j;

  fd = open("example.encoded", O_WRONLY|O_CREAT|O_TRUNC, 0644);
  encoder = labcomm_encoder_new(labcomm_fd_writer, &fd);
  labcomm_encoder_register_example_log_message(encoder);
  labcomm_encoder_register_example_data(encoder);
  for (i = 0 ; i < argc ; i++) {
    example_log_message message;

    message.sequence = i + 1;
    message.line.n_0 = i;
    message.line.a = malloc(message.line.n_0*sizeof(message.line));
    for (j = 0 ; j < i ; j++) {
      message.line.a[j].last = (j == message.line.n_0 - 1);
      message.line.a[j].data = argv[j + 1];
    }
    labcomm_encode_example_log_message(encoder, &message);
    free(message.line.a);
  }
  for (i = 0 ; i < argc ; i++) {
    float f = i;
    labcomm_encode_example_data(encoder, &f);
  }
}
\end{verbatim}

Running \verb+./example_encoder one two+, will yield the following result in example.encoded:


\begin{verbatim}
00000000  02 40 0b 6c 6f 67 5f 6d  65 73 73 61 67 65 11 02  |.@.log_message..|
00000010  08 73 65 71 75 65 6e 63  65 23 04 6c 69 6e 65 10  |.sequence#.line.|
00000020  01 00 11 02 04 6c 61 73  74 20 04 64 61 74 61 27  |.....last .data'|
00000030  02 41 04 64 61 74 61 25  40 00 00 00 01 00 40 00  |.A.data%@.....@.|
00000040  00 00 02 01 01 03 6f 6e  65 40 00 00 00 03 02 00  |......one@......|
00000050  03 6f 6e 65 01 03 74 77  6f 41 00 00 00 00 41 3f  |.one..twoA....A?|
00000060  80 00 00 41 40 00 00 00                           |...A@...|
00000068
\end{verbatim}

i.e.,
\begin{verbatim}
<sample_decl> <user_id: 0x40> <string: <len: 11> <"log_message"> 
  <struct_decl: 
    <number_of_fields: 2>
    <string: <len: 8> <"sequence"> <type: <integer_type>>
    <string: <len: 4> <"line">> <type: <array_decl 
      <number_indices: 1> <variable_index> 
      <type: <struct_decl: 
        <number_of_fields:2>
        <string: <len: 4> <"last">> <type: <boolean_type>> 
        <string: <len: 4> <"data">> <type: <string_type>>
      >>
   >
>
<sample_decl> <user_id: 0x41>
   ...
\end{verbatim}

\section{Ideas/Discussion}:

The labcomm language is more expressive than its target languages regarding data types.
E.g., labcomm can declare both arrays of arrays and matries where Java only has arrays of arrays
In the generated Java code, a labcomm matrix is implemented as an array of arrays.

Another case (not yet included) is unsigned types, which Java doesn't have. If we include
unsigned long in labcomm, that has a larger range of values than is possible to express using
Java primitive types. However, it is unlikely that the entire range is actually used, so one
way of supporting the common cases is to include run-time checks for overflow in the Java encoders
and decoders.

\appendix
\newpage
\section{The LabComm language}
\label{sec:LanguageGrammar}

\subsection{Abstract syntax}
\begin{verbatim}
Program ::= Decl*;

abstract Decl ::= Type <Name:String>;
TypeDecl : Decl;
SampleDecl : Decl;

Field ::= Type <Name:String>;

abstract Type;
VoidType          : Type;
PrimType          : Type ::= <Name:String> <Token:int>;
UserType          : Type ::= <Name:String>;
StructType        : Type ::= Field*;
ParseArrayType    : Type ::= Type Dim*;
abstract ArrayType : Type ::= Type Exp*;
VariableArrayType : ArrayType;
FixedArrayType    : ArrayType;

Dim ::= Exp*;

abstract Exp;
IntegerLiteral : Exp ::= <Value:String>;
VariableSize : Exp;
\end{verbatim}

\newpage
\section{The LabComm protocol}
\label{sec:ProtocolGrammar}

\begin{verbatim}
<packet> := ( <type_decl> | <sample_decl> | <sample_data> )*
<type_decl> := 0x01 ''(packed)'' <user_id> <string> <type>
<sample_decl> := 0x02 ''(packed)''<user_id> <string> <type>
<user_id> := 0x40..0xffffffff  ''(packed)''
<string> := <string_length> <char>*
<string_length> := 0x00..0xffffffff  ''(packed)''
<char> := any UTF-8 char
<type> := ( <basic_type> | <user_id> | <array_decl> | <struct_decl> )
<basic_type> := ( <boolean_type> | <byte_type> | <short_type> |
                  <integer_type> | <long_type> | <float_type> | 
                  <double_type> | <string_type> )
<boolean_type> := 0x20 ''(packed)''
<byte_type> := 0x21 ''(packed)''
<short_type> := 0x22 ''(packed)''
<integer_type> := 0x23 ''(packed)''
<long_type> := 0x24 ''(packed)''
<float_type> := 0x25 ''(packed)''
<double_type> := 0x26 ''(packed)''
<string_type> := 0x27 ''(packed)''
<array_decl> := 0x10 ''(packed)'' <number_of_indices> <indices> <type>
<number_of_indices> := 0x00..0xffffffff  ''(packed)''
<indices> := ( <variable_index> | <fixed_index> )*
<variable_index> := 0x00  ''(packed)''
<fixed_index> := 0x01..0xffffffff  ''(packed)''
<struct_decl> := 0x11 ''(packed)'' <number_of_fields> <field>*
<number_of_fields> := 0x00..0xffffffff  ''(packed)''
<field> := <string> <type>
<sample_data> := <user_id> <packed_sample_data>
<packed_sample_data> := is sent in network order, sizes are as follows:

||Type           ||Encoding/Size                                         ||
||---------------||------------------------------------------------------||
||boolean        ||  8 bits                                              ||
||byte           ||  8 bits                                              ||
||short          || 16 bits                                              ||
||integer        || 32 bits                                              ||
||long           || 64 bits                                              ||
||float          || 32 bits                                              ||
||double         || 64 bits                                              ||
||string         || length ''(packed)'', followed by UTF8 encoded string ||
||array          || each variable index ''(packed)'',                    ||
||               || followed by encoded elements                         ||
||struct         || concatenation of encoding of each element            ||
||               || in declaration order                                 ||
\end{verbatim}

Type fields, user IDs, number of indices and lengths are sent in a packed, or
variable length, form.  An integer is sent as a sequence of bytes where the
lower seven bits contain a chunk of the actual number and the high bit
indicates if more chunks follow. The sequence of chunks are sent with the least
significant chunk first.  (The numbers encoded in this form are indicated above
with \textit{(packed)}.)

\end{document}
