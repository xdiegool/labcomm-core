% *** en embryo of a technical report describing the labcomm design rationale and implementation ***

\documentclass[a4paper]{article}
\usepackage{listings}
%\usepackage{verbatims}
%\usepackage{todo}

\begin{document}
\title{Labcomm tech report}
\author{Anders Blomdell and Sven Gesteg\aa{}rd Robertz }
\date{draft, \today}

\maketitle

\begin{abstract}

LabComm is a binary protocol suitable for transmitting and storing samples of
process data. It is self-describing and independent of programming language,
processor, and network used (e.g., byte order, etc).  It is primarily intended
for situations where the overhead of communication has to be kept at a minimum,
hence LabComm only requires one-way communication to operate. The one-way
operation also has the added benefit of making LabComm suitable as a storage
format.

LabComm provides self-describing channels, as communication starts with the
transmission of an encoded description of all possible sample types that can
occur, followed by any number of actual samples in any order the sending
application sees fit.

The LabComm system is based on a binary protocol and
and a compiler that generates encoder/decoder routines for popular languages
including C, Java, and Python. There is also a standard library for the
languages supported by the compiler, containing generic routines for
encoding and decoding types and samples, and interaction with
application code.

The LabComm compiler accepts type and sample declarations in a small language
that is similar to C or Java type-declarations.
\end{abstract}
\section{Introduction}

%[[http://rfc.net/rfc1057.html|Sun RPC]]
%[[http://asn1.org|ASN1]].

LabComm has got it's inspiration from Sun RPC~\cite{SunRPC}
and ASN1~\cite{ASN1}. LabComm is primarily intended for situations
where the overhead of communication has to be kept at a minimum, hence LabComm
only requires one-way communication to operate. The one-way operation also has
the added benefit of making LabComm suitable as a storage format.

Two-way comminication adds complexity, in particular for hand-shaking
during establishment of connections, and the LabComm library provides
support for (for instance) avoiding deadlocks during such hand-shaking.

\pagebreak
\section{Communication model}

LabComm provides self-describing communication channels, by always transmitting
a machine readable description of the data before actual data is
sent\footnote{Sometimes referred to as \emph{in-band} self-describing}.
Therefore, communication on a LabComm channel has two phases

\begin{enumerate}
\item the transmission of signatures (an encoded description including data
types and names, see appendix~\ref{sec:ProtocolGrammar} for details) for all sample types
that can be sent on the channel
\item the transmission of any number of actual samples in any order
\end{enumerate}

During operation, LabComm will ensure (i.e., monitor) that a communication
channel is fully configured, meaning that both ends agree on what messages that
may be passed over that channel.  If an unregistered sample type is sent or
received, the LabComm encoder or decoder will detect it and take action.
In more dynamic applications, it is possible to reconfigure a channel in order to add,
remove, or change the set of registered sample types. This is discussed
in Section~\ref{sec:reconfig}.

The roles in setting up, and maintaining, the configuration of a channel are as follows:

\paragraph{The application software} (or higher-level protocol) is required to

\begin{itemize}
\item register all samples to be sent on a channel with the encoder
\item register handlers for all samples to be received  on a channel with the decoder
\end{itemize}

\paragraph{The transmitter (encoder)}

\begin{itemize}
 \item ensures that the signature of a sample is transmitted on the channel before samples are
       written to that channel
\end{itemize}

\paragraph{The receiver (decoder)}

\begin{itemize}
 \item checks, for each signature, that the application has registered a handler for that sample type
 \item if an unhandled signature is received, pauses the channel and informs the application
\end{itemize}

The fundamental communication model applies to all LabComm channels and
deals with the individual unidirectional channels. In addition to that,
the labcomm libraries support the implementation of higher-level
handshaking and establishment of bidirectional channels both through
means of interacting with the underlying transport layer (e.g., for
marking packets containing signatures as \emph{important}, for
transports that handle resending of dropped packets selectively), or
requesting retransmission of signatures.

In order to be both lean and generic, LabComm does not provide a
complete protocol for establishing and negotiating bidirectional
channels, but does provide support for building such protocols on top
of LabComm.
\subsection{Reconfiguration}
\label{sec:reconfig}

The fundamental communication model can be generalized to the life-cycle
of a concrete communication channel, including the transport layer,
between two end-points. Then, the communication phases are
\begin{enumerate}
  \item \emph{Establishment} of communication channel at the transport layer
  \item \emph{Configuration} of the LabComm channel (registration of sample
    types)
  \item \emph{Operation} (transmission of samples)
\end{enumerate}
where it is possible to \emph{reconfigure} a channel by transitioning
back from phase 3 to phase 2. That allows dynamic behaviour, as a sample
type can be added or redefined at run-time. It also facilitates error
handling in two-way channels.

One example of this, more dynamic, view of a labcomm channel is that the
action taken when an unregistered sample is sent or received is to
revert back to the configuration phase and redo the handshaking to
ensure that both sides agree on the set of sample types (i.e.,
signatures) that are currently configured for the channel.

From the system perspective, the LabComm protocol is involved in
phases 2 and 3. The establishement of the \emph{transport-layer}
channels is left to external application code. In the Labcomm library,
that application code is connected to the LabComm routines through
the \emph{reader} and \emph{writer} interfaces,
with default implementations for sockets or file descriptors (i.e.,
files and streams).


\section{The Labcomm language}

The LabComm language is used to describe data types, and from such data
descriptions, the compiler generates code for encoding and decoding
samples. The language is quite similar to C struct declarations, with
some exceptions. We will now introduce the language through a set of
examples.

These examples do not cover the entire language
specification (see appendix~\ref{sec:LanguageGrammar} for the complete
grammar), but serve as a gentle introduction to the LabComm
language covering most common use-cases.

\subsection{Primitive types}

\begin{verbatim}
  sample boolean a_boolean;
  sample byte a_byte;
  sample short a_short;
  sample int an_int;
  sample long a_long;
  sample float a_float;
  sample double a_double;
  sample string a_string;
\end{verbatim}

\subsection{The void type}

There is a type, \verb+void+, which can be used to send
a sample that contains no data. 

\begin{verbatim}
typedef void an_empty_type;

sample an_empty_type no_data1;
sample void no_data2;
\end{verbatim}

\verb+void+ type can may not be used as a field in a struct or
the element type of an array.


\subsection{Arrays}

\begin{verbatim}
  sample int fixed_array[3];
  sample int variable_array[_];                // Note 1
  sample int fixed_array_of_array[3][4];       // Note 2
  sample int fixed_rectangular_array[3, 4];    // Note 2
  sample int variable_array_of_array[_][_];    // Notes 1 & 2
  sample int variable_rectangular_array[_, _]; // Notes 1 & 2
\end{verbatim}

\begin{enumerate}
\item In contrast to C, LabComm supports both fixed and variable (denoted
by~\verb+_+) sized arrays.

\item In contrast to Java, LabComm supports multidimensional arrays and not
only arrays of arrays.

\end{enumerate}

\subsection{Structures}

\begin{verbatim}
  sample struct {
    int an_int_field;
    double a_double_field;
  } a_struct;
\end{verbatim}

\subsection{Sample type refereces}

In addition to the primitive types, a sample may contain
a reference to a sample type. References are declared using
the \verb+sample+ keyword.

Examples:

\begin{verbatim}
sample sample a_ref;

sample sample ref_list[4];

sample struct { 
    sample ref1;
    sample ref2;
    int x;
    int y;
} refs_and_ints;
\end{verbatim}

Sample references are need to be registered on both encoder and decoder
side, using the functions

\begin{verbatim}
int labcomm_decoder_sample_ref_register(
    struct labcomm_decoder *decoder\nonumber
    const struct labcomm_signature *signature);

int labcomm_encoder_sample_ref_register(
    struct labcomm_encoder *encoder\nonumber
    const struct labcomm_signature *signature);
\end{verbatim}

The value of an unregistered sample reference will be decoded as \verb+null+.

\subsection{User defined types}

User defined types are declared with the \verb+typedef+ reserved word,
and can then be used in type and sample declarations.

\begin{verbatim}
  typedef struct {
    int field_1;
    byte field_2;
  } user_type[_];
  sample user_type a_user_type_instance;
  sample user_type another_user_type_instance;
\end{verbatim}

\section{The LabComm system}

The LabComm system consists of a compiler for generating code from the data
descriptions, and libraries providing LabComm communication facilities in,
currently, C, Java, Python, C\#, and RAPID\footnote{excluding variable
size samples, as RAPID has limited support for dynamic memory allocation}.


\subsection{The LabComm compiler}

The LabComm compiler generates code for the declared samples, including marshalling and
demarshalling code, in the supported target languages.

The compiler itself is implemented in Java using the JastAdd~\cite{jastadd} compiler compiler.

\subsection{The LabComm library}

The LabComm libraries contain functionality for the end-to-end transmission
of samples. They are divided into two layers, where the upper layer implements
the general encoding and decoding of samples, and the lower one deals with
the transmission of the encoded byte stream on a particular transport layer.

Thus, the LabComm communication stack looks like this:
\begin{figure}[h!]
\begin{verbatim}
    _______________________
    |     Application     |
    +---------------------+
    | encoder  | decoder  |    to/from labcomm encoded byte stream
    +----------+----------+
    | writer   | reader   |    transmit byte stream over particular transport
    +----------+----------+
    | transport layer / OS|
    +---------------------+
\end{verbatim}
\end{figure}
\subsubsection{LabComm actions}

(similar to ioctl())
The encoder/writer and decoder/reader interfaces consist of a set of actions

One example of this is that there is a a separate writer action for
transmitting signatures, allowing the writer to treat a signature differently
from encoded samples, e.g., to allow handshaking during channel setup.

User actions allow the application or a higher level
protocol to communicate with the underlying transport layer through the LabComm
encoder.

One example is reliable communication, which is controlled from the application
but needs to be implemented for each transport at at the reader/writer level.
(Or not, e.g., TCP)

\section{LabComm is not...}

\begin{itemize}
\item a protocol for two-way connections
\item intrinsically supporting reliable communication
\item providing semantic service-descriptions
\end{itemize}

But

\begin{itemize}
\item it is suitable for the individual channels of a structured connection
\item the user action mechanism allows using feature of different transport layers
  through labcomm (i.e., it allows encapsulation of the transport layer)
\item the names of samples can be chosen and mapped according to a suitable taxonomy or ontology
\end{itemize}



\section{Example and its encoding}

With the following `example.lc` file:

\lstinputlisting[basicstyle=\footnotesize\ttfamily]{../examples/wiki_example/example.lc}
and this \verb+example_encoder.c+ file
\lstinputlisting[basicstyle=\footnotesize\ttfamily,language=C]{../examples/wiki_example/example_encoder.c}

\newpage

Running \verb+./example_encoder one two+, will yield the following result in example.encoded:
\begin{verbatim}
00000000  01 0c 0b 4c 61 62 43 6f  6d 6d 32 30 31 34 02 30 |...LabComm2014.0|
00000010  40 0b 6c 6f 67 5f 6d 65  73 73 61 67 65 22 11 02 |@.log_message"..|
00000020  08 73 65 71 75 65 6e 63  65 23 04 6c 69 6e 65 10 |.sequence#.line.|
00000030  01 00 11 02 04 6c 61 73  74 20 04 64 61 74 61 27 |.....last .data'|
00000040  02 08 41 04 64 61 74 61  01 25 40 04 00 00 00 01 |..A.data.%@.....|
00000050  00 40 09 00 00 00 02 01  01 03 6f 6e 65 40 0e 00 |.@........one@..|
00000060  00 00 03 02 00 03 6f 6e  65 01 03 74 77 6f 41 04 |......one..twoA.|
00000070  00 00 00 00 41 04 3f 80  00 00 41 04 40 00 00 00 |....A.?...A.@...|
00000080
\end{verbatim}

i.e.,
\begin{verbatim}
<version> <length: 12> <string: <len: 11> <"LabComm2014">>
<sample_decl> <length: 48 
              <user_id: 0x40> 
              <string: <len: 11> <"log_message">
  <signature_length: 34>
  <struct_decl:
    <number_of_fields: 2>
    <string: <len: 8> <"sequence"> <type: <integer_type>>
    <string: <len: 4> <"line">> <type: <array_decl
      <number_indices: 1> <variable_index>
      <type: <struct_decl:
        <number_of_fields:2>
        <string: <len: 4> <"last">> <type: <boolean_type>>
        <string: <len: 4> <"data">> <type: <string_type>>
      >>
   >
>
<sample_decl> <length: 8> <user_id: 0x41> <string: <len: 4> <"data">>
  <signature_length: 1> <float_type>
<sample_data> <user_id: 40> <length: 4>  <packed_sample_data>
<sample_data> <user_id: 40> <length: 9>  <packed_sample_data>
<sample_data> <user_id: 40> <length: 14>  <packed_sample_data>
\end{verbatim}

\section{Type and sample declarations}

LabComm has two constructs for declaring sample types, \emph{sample
declarations} and \emph{type declarations}. A sample declaration is used
for the concrete sample types that may be transmitted, and is always
encoded as a \emph{flattened} signature. That means that a sample
containing user types, like

\begin{verbatim}
typedef struct {
  int x;
  int y;
} point;

sample struct {
  point start;
  point end;
} line;
\end{verbatim}

is flattened to 

\begin{verbatim}
sample struct {
  struct {
    int x;
    int y;
  } start;
  struct {
    int x;
    int y;
  } end;
} line;
\end{verbatim}

Sample declarations are always sent, and is the fundamental identity of
a type in LabComm. 

Type declarations is the hierarchical counterpart to sample
declarations: here, fields of user types are encoded as a reference to
the type instead of being flattened. As the flattened sample decl is the
fundamental identity of a type, type declarations can be regarded as
meta-data, describing the internal structure of a sample. They are
intended to be read by higher-level software and human system developers
and integrators.

Sample declarations and type declarations have separate name-spaces in
the sense that the numbers assigned to them by a labcomm encoder 
come from two independent number series. To identify which
\verb+TYPE_DECL+ a particular \verb+SAMPLE_DECL+ corresponds to, the
\verb+TYPE_BINDING+ packet is used.

For sample types that do not depend on any typedefs, no \verb+TYPE_DECL+
is sent, and the \verb+TYPE_BINDING+ binds the sample id to the special
value zero to indicate that the type definition is identical to the
flattened signature.

\subsection{Example}

The labcomm declaration
\lstinputlisting[basicstyle=\footnotesize\ttfamily]{../examples/user_types/test.lc}
can be is encoded as
\begin{lstlisting}[basicstyle=\footnotesize\ttfamily]
TYPE_DECL 0x40 "coord" <int> val
TYPE_DECL 0x41 "point" <struct> <2 fields> 
                                "x" <type: 0x40> 
                                "y" <type: 0x40>
TYPE_DECL 0x42 "line" <struct> <2 fields> 
                                "start" <type: 0x41> 
                                "end" <type: 0x41>
TYPE_DECL 0x43 "foo" <struct> <3 fields> 
                                "a" <int> 
                                "b" <int> 
                                "c" <boolean>
TYPE_DECL 0x44 "twolines" <struct> <3 fields> 
                                "l1" <type:0x42> 
                                "l2" <type:0x42> 
                                "f" <type:0x43>

SAMPLE_DECL 0x40 "twolines" <flat signature>

TYPE_BINDING 0x40 0x44
\end{lstlisting}

Note that the id 0x40 is used both for the \verb+TYPE_DECL+ of
\verb+coord+ and the \verb+SAMPLE_DECL+ of \verb+twoline+, and that the
\verb+TYPE_BINDING+ binds the sample id \verb+0x40+ to the type id
\verb+0x44+.

\subsection{Run-time behaviour}

When a sample type is registered on an encoder, a \verb+SAMPLE_DECL+
(i.e., the flat signature) is always generated on that encoder channel.

If the sample depends on user types (i.e., typedefs), \verb+TYPE_DECL+
packets are encoded, recursively, for the dependent types and a 
corresponding \verb+TYPE_BINDING+ is encoded.

If a \verb+TYPE_DECL+ is included via multiple sample types, or
dependency paths, an encoder may choose to only encode it once, but is
not required to do so. However, if multiple \verb+TYPE_DECL+ packets are
sent for the same \verb+typedef+, the encoder must use the same
\verb+type_id+.

\subsection{Decoding in-band type descriptions}

In LabComm, the in-band data descriptions are equivalent to \footnote{in
the sense that they contain all information needed to recreate} the data
description source (i.e., the ``.lc-file'').
%
As the type declarations (a.k.a. \emph{signatures}) are written before
sample data on every channel, they can be used to interpret data with
an unknown (by the receiver) format.

The libraries provide functionality to subscribe to (i.e., register a
\emph{handler} for) sample and type declarations.

On the low level, the handler receives an instance of the signature
data structure corresponding to the received declaration.


For higher-level processing, the Java library provides the
\verb+ASTbuilder+ class, which builds an abstract syntax tree in
the internal representation of the LabComm compiler.
That enables the user to use the complete functionality of the
LabComm compiler, e.g. code generation,  on declarations received in a
LabComm stream. 


In combination with on-the-fly compilation and class-loading (or
linking) that makes it possible to dynamically create handlers for
previously unknown data types. Thereby, it enables dynamic configuration
of LabComm endpoints in static languages without the overhead of
interpreting signatures (at the one-time cost of code generation and
compilation).







\section{Ideas/Discussion}:

The labcomm language is more expressive than its target languages regarding data types.
E.g., labcomm can declare both arrays of arrays and matries where Java only has arrays of arrays
In the generated Java code, a labcomm matrix is implemented as an array of arrays.

Another case (not yet included) is unsigned types, which Java doesn't have. If we include
unsigned long in labcomm, that has a larger range of values than is possible to express using
Java primitive types. However, it is unlikely that the entire range is actually used, so one
way of supporting the common cases is to include run-time checks for overflow in the Java encoders
and decoders.

\section{Related work}
  
Two in-band self-descibing communication protocols are Apache
Avro\cite{avro} and EDN, the extensible data notation developed for
Clojure and Datomic\cite{EDN}.

EDN encodes \emph{values} as UTF-8 strings. The documentation says
``edn is a system for the conveyance of values. It is not a type system,
and has no schemas.'' That said, it is \emph{extensible} in the sense
that it has a special \emph{dispatch charachter}, \verb+#+, which can  
be used to add a \emph{tag} to a value. A tag indicates a semantic
interpretation of a value, and that allows the reader to support
handlers for specific tags, enabling functionality similar to that of
labcomm.

\subsection{Apache Avro}

Apache Avro is similar to LabComm in that it has a textual language
for declaring data, a binary protocol for transmitting data, and code
generation for several languages.

Avro is a larger system, including RPC \emph{protocols}, support for
using different \emph{codecs} for data compression, and \emph{schema
resolution} to support handling schema evolution and transparent 
interoperability between different versions of a schema.

\subsubsection*{Data types} 

In the table, the Avro type names are listed, and matched to the
corresponding LabComm type:

\begin{tabular}{|l|c|c|}
\hline
  Type &            Labcomm  &               Avro \\
  \hline Primitive types \\ \hline

int    &         4 bytes     &           varint  \\
long   &         8 bytes     &           varint  \\
float  &         4 bytes     &           4 bytes \\
long   &         8 bytes     &           8 bytes \\
string &         varint + utf8[]   &     varint + utf8[] \\ 
bytes  &         varint + byte[]   &     varint + byte[]\\

  \hline Complex types  \\ \hline

struct/record &  concat of fields     &  concat of fields \\ 
arrays        &  varIdx[] : elements  &  block[]          \\
map           &    n/a                &  block[]          \\
union         &   n/a                 & (varint idx) : value \\
fixed         &   byte[n]             &  the number of bytes declared in
the schema\\
\hline
\end{tabular}

  where 

\begin{verbatim}  
  block ::= (varint count) : elem[count]      [*1]
  count == 0 --> no more blocks


[*1] for arrays, count == 0 --> end of array
     if count < 0, there are |count| elements
     preceded by a varint block_size to allow
     fast skipping
\end{verbatim}  

In maps, keys are strings, and values  according to the schema.

In unions, the index indicates the kind of value and the
value is encoded according to the schema.

Note that the Avro data type \verb+bytes+ corresponds to the
LabComm declaration \verb+byte[_]+, i.e. a varaible length byte array.

\subsubsection*{the wire protocol}

\begin{tabular}{|l|c|c|}
  \hline
  What & LabComm & Avro \\ \hline
  Data description & Binary signature & JSON schema \\
  Signature sent only once pre connection& posible & possible \\
  Signature sent with each sample & possible & possible \\
  Data encoding & binary & binary \\
  \hline
\end{tabular}


Both avro and labcomm use varints when encoding data, similar in that
they both send a sequence of bytes containing 7 bit chunks (with the
eight bit signalling more chunks to come), but they differ in range,
endianness and signedness.

\begin{verbatim}
                LabComm                 Avro
                unsigned 32 bit         signed zig-zag coding
                most significant chunk  least significant chunk
                first                   first

                0   ->  00               0  ->  00
                1   ->  01              -1  ->  01
                2   ->  02               1  ->  02
                    ...                 -2  ->  03
                                         2  ->  04
                                            ...
                127 ->  7f              -64 ->  7f
                128 ->  81 00            64 ->  80 01
                129 ->  81 01           -65 ->  81 01
                130 ->  81 02            65 ->  82 01
                    ...                     ...   
\end{verbatim}

\paragraph{Avro Object Container Files} can be seen as a counterpart
  to a LabComm channel: 
Avro includes a simple object container file format. A file has a
schema, and all objects stored in the file must be written according to
that schema, using binary encoding. Objects are stored in blocks that
may be compressed. Syncronization markers are used between blocks to
permit efficient splitting of files, and enable detection of 
corrupt blocks.


The major difference is the sync markers that LabComm does not have, as
LabComm assumes that, while the transport may drop packets, there will
be no bit errors in a received packet. If data integrity is required,
that is delegated to the reader and writer for the particular transport.

\subsubsection{Representation of hierarchical data types}

For a type that contains fields of other user types, like
\begin{verbatim}
typedef struct {
  int x;
  int y;
} Point;

sample struct {
  Point start;
  Point end;
} line;
\end{verbatim}

LabComm encodes both the flattened signature and the 
typedef which allows the hierarchical type structure to be
reconstructed.
%
The avro encoding is quite similar. 
The \verb+Line+ example, corresponds to the two schemas
\begin{verbatim}
{"namespace": "example.avro",
 "type": "record",
 "name": "Point",
 "fields": [
     {"name": "x", "type": "int"},
     {"name": "y", "type": "int"}
 ]
}
\end{verbatim}
and
\begin{verbatim}
{"namespace": "example.avro",
 "type": "record",
 "name": "Line",
 "fields": [
     {"name": "start", "type": "Point"},
     {"name": "end",   "type": "Point"}
 ]
}
\end{verbatim}
which is encoded in an Object Container File as
\begin{verbatim}
{"type":"record",
 "name":"Line",
 "namespace":"example.avro",
 "fields":[{"name":"start",
            "type":{"type":"record",
                    "name":"Point",
                    "fields":[{"name":"x","type":"int"},
                              {"name":"y","type":"int"}]}},
           {"name":"end",
            "type":"Point"}
 ]
}
\end{verbatim}
\subsubsection{Fetures not in LabComm} 

Avro has a set of features with no counterpart in LabComm. They include

\paragraph{Codecs.}

Avro has multiple codecs (for compression of the data):

    \begin{verbatim}
    Required Codecs:
    - null : The "null" codec simply passes through data uncompressed.

    - deflate : The "deflate" codec writes the data block using the deflate
                algorithm as specified in RFC 1951, and typically implemented using the
                zlib library. Note that this format (unlike the "zlib format" in RFC
                1950) does not have a checksum.

    Optional Codecs

    - snappy:   The "snappy" codec uses Google's Snappy compression library. Each
                compressed block is followed by the 4-byte, big-endian CRC32 checksum of
                the uncompressed data in the block.

    \end{verbatim}

  \paragraph{Schema Resolution.} The main objective of LabComm is to
    ensure correct operation at run-time. Therefore, a LabComm decoder
    requires the signatures for each handled sample to match exactly.

    Avro, on the other hand, supports the evolution of schemas and
    provides support for reading data where the ordering of fields
    differ (but names and types are the same), numerical types differ
    but can be
    \emph{promoted} (E.g., \verb+int+ can be promoted to \verb+long+,
    \verb+float+, or \verb+double+.), and record fields have been added
    or removed (but are nullable or have default values).

    \paragraph{Schema fingerprints.} Avro defines a \emph{Parsing
    Canonical Form} to define when two JSON schemas are ``the same''.
    To reduce the overhead when, e.g., tagging data with the schema
    there is support for creating a \emph{fingerprint} using 64/128/256
    bit hashing, in combination with a centralized repository for
    fingerprint/schema pairs.

\bibliography{refs}{}
\bibliographystyle{plain}

\appendix
\newpage

\section{The LabComm language}
\label{sec:LanguageGrammar}

\subsection{Abstract syntax}
\begin{verbatim}
Program ::= Decl*;

abstract Decl ::= Type <Name:String>;
TypeDecl : Decl;
SampleDecl : Decl;

Field ::= Type <Name:String>;

abstract Type;
VoidType          : Type;
SampleRefType     : Type;
PrimType          : Type ::= <Name:String> <Token:int>;
UserType          : Type ::= <Name:String>;
StructType        : Type ::= Field*;
ParseArrayType    : Type ::= Type Dim*;
abstract ArrayType : Type ::= Type Exp*;
VariableArrayType : ArrayType;
FixedArrayType    : ArrayType;

Dim ::= Exp*;

abstract Exp;
IntegerLiteral : Exp ::= <Value:String>;
VariableSize : Exp;
\end{verbatim}

\newpage
\section{The LabComm protocol}
\label{sec:ProtocolGrammar}

Each LabComm2014 packet has the layout
\begin{verbatim}
<id> <length> <data...>
\end{verbatim}
where \verb+length+ is the number of bytes of the \verb+data+ part
(i.e., excluding the \verb+id+ and \verb+length+ fields), and 
the \verb+id+ gives the layout of the \verb+data+ part as defined 
in \ref{sec:ConcreteGrammar}
\subsection{Data encoding}
LabComm primitive types are encoded as fixed width values, sent in
network order.  Type fields, user IDs, number of indices and lengths are
sent in a variable length (\emph{varint}) form.  A varint integer value
is sent as a sequence of bytes where the lower seven bits contain a
chunk of the actual number and the high bit indicates if more chunks
follow. The sequence of chunks are sent with the least significant chunk
first.  

The built-in data types are encoded as follows:
\begin{lstlisting}[basicstyle=\footnotesize\ttfamily]
||Type       ||Encoding/Size                                      ||
||-----------||---------------------------------------------------||
||boolean    ||  8 bits                                           ||
||byte       ||  8 bits                                           ||
||short      || 16 bits                                           ||
||integer    || 32 bits                                           ||
||long       || 64 bits                                           ||
||float      || 32 bits                                           ||
||double     || 64 bits                                           ||
||sample_ref || 32 bits                                           ||
||string     || length (varint), followed by UTF8 encoded string  ||
||array      || each variable index (varint),                     ||
||           || followed by encoded elements                      ||
||struct     || concatenation of encoding of each element         ||
||           || in declaration order                              ||
\end{lstlisting}
\pagebreak
\subsection{Protocol grammar}
\label{sec:ConcreteGrammar}
\begin{lstlisting}[basicstyle=\footnotesize\ttfamily]
<packet>          := <id> <length> ( <version>      | 
                                     <type_decl>    | 
                                     <sample_decl>  |
                                     <sample_ref>   |
                                     <type_binding> |
                                     <sample_data> )
<version>         := <string>
<sample_decl>     := <sample_id> <string> <signature>
<sample_ref>      := <sample_id> <string> <signature>
<type_decl>       := <type_id> <string> <signature>
<type_binding>    := <sample_id> <type_id>
<user_id>         := 0x40..0xffffffff  
<sample_id> : <user_id>
<type_id>   : <user_id>
<string>          := <string_length> <char>*
<string_length>   := 0x00..0xffffffff  
<char>            := any UTF-8 char
<signature>       := <length> <type>
<type>            := <length> ( <basic_type>  | 
                                <array_decl>  | 
                                <struct_decl> | 
                                <type_id> )
<basic_type>      := ( <void_type> | <boolean_type> | <byte_type> | <short_type> |
                     <integer_type> | <long_type> | <float_type> |
                     <double_type> | <string_type> | <sample_ref_type>)

<void_type>       := <struct_decl> 0 //void is encoded as empty struct
<boolean_type>    := 0x20 
<byte_type>       := 0x21 
<short_type>      := 0x22 
<integer_type>    := 0x23 
<long_type>       := 0x24 
<float_type>      := 0x25 
<double_type>     := 0x26 
<string_type>     := 0x27 
<sample_ref_type> := 0x28 
<array_decl>      := 0x10  <nbr_of_indices> <indices> <type>
<nbr_of_indices>  := 0x00..0xffffffff  
<indices>         := ( <variable_index> | <fixed_index> )*
<variable_index>  := 0x00  
<fixed_index>     := 0x01..0xffffffff  
<struct_decl>     := 0x11  <nbr_of_fields> <field>*
<nbr_of_fields>   := 0x00..0xffffffff  
<field>           := <string> <type>
<sample_data>     := packed sample data sent in network order, with
                  primitive type elements encoded according to
                  the sizes above
\end{lstlisting}
where the \verb+<id>+ in \verb+<packet>+ signals the type of payload,
and may be either a \verb+<sample_id>+ or a system packet id.
The labcomm sytem packet ids are:
\begin{lstlisting}[basicstyle=\footnotesize\ttfamily]
version:      0x01 
sample_decl:  0x02 
sample_ref:   0x03 
type_decl:    0x04 
type_binding: 0x05          
\end{lstlisting}
Note that since the signature transmitted in a \verb+<sample_def>+ is
flattened, the \verb+<type>+ transmitted in a \verb+<sample_def>+ may
not contain any \verb+<type_id>+ fields.
\end{document}
